\documentclass[preprint]{sigplanconf}

\usepackage{amsmath}
\usepackage{amssymb}
\usepackage{natbib}
\usepackage{multirow}
\usepackage{setspace}

\include{paper}

%include paper.fmt
%format +|+ = "\mathbin{\ddagger}"
%format &+ = "\mathbin{\cup}"
%format &* = "\mathbin{\cap}"
%format `elem_` = "\mathbin{\overline{\in}}"
%format hole = bullet

%format let_ = "\mathsf{let}"
%format case_ = "\mathsf{case}"
%format bot = "\bot{}"

%format (ELam a (b)) = "\ll{}" \a -> b "\rr{}"
%format (EFun a b) = "\ll{}" a \? b "\rr{}"
%format (ECon a b) = "\ll{}" a \? b "\rr{}"
%format (EApp a b) = "\ll{}" a \? b "\rr{}"
%format (EVar a) = "\ll{}" a "\rr{}"
%format (ECase a b) = "\ll{}\!\!" case a of b "\rr{}"
%format (ELet a b c) = "\ll{}\!\!" let a = b in c "\rr{}"
%format (EAlt a b c) = "\ll{}" a \? b -> c "\rr{}"

\newcommand{\simp}[2]{\begin{minipage}{6.4cm}#2\end{minipage} & (#1) \\}
\newenvironment{simplify}
    {\noindent
     \begin{flushright}
     \begin{tabular}{p{6.5cm}r}
    }
    {\end{tabular}
     \end{flushright}
    }

\newenvironment{definition}
    {\smallskip
     \noindent\textbf{Definition:}}
    {\noexample}

\newenvironment{lemma}[1]
    {\smallskip
     \noindent\textbf{Lemma:} \textit{#1}}
    {\noexample}

\hsdef{\begin{comment}
Show
f,x,y,g,v,e,s_0,l',b',n,l_n,s,l_n',xs,b_n',e_n,e_6
lp,p,bp,l,b,xs_
app,lam,fun,con
\end{comment}}
\begin{comment}
\begin{code}
import Prelude hiding (map)
import Data.List hiding (map)

data Expr = EVar String
          | ECon String [Expr]
          | EFun String [Expr]
          | EApp Expr [Expr]
          | ELam String Expr
          | ELet String Expr Expr
          | ECase Expr [Alt]
data Alt = EAlt String [String] Expr

instance Eq Expr

type FuncName = String
body   :: FuncName  -> Expr
rhs    :: Alt       -> Expr

data World = World
expensive :: a -> b
primEqInt, primNeqInt :: Int -> Int -> Bool
primCatch :: alpha -> (IOError -> alpha) -> alpha
apply :: a
ellipses :: a
hole :: a
bottom :: a

type VarName   = String
freeVars :: Expr -> [VarName]
universe :: Expr -> [Expr]
transform :: (Expr -> Expr) -> Expr -> Expr
\end{code}
\end{comment}

\begin{document}

\conferenceinfo{ICFP 2009}{date, City.} %
\copyrightyear{2009} %
\copyrightdata{[to be supplied]}

\titlebanner{} % \today{} - \currenttime{}}        % These are ignored unless
\preprintfooter{}   % 'preprint' option specified.

\title{Losing Functions without Gaining Data}
\subtitle{ -- a new method for defunctionalisation}

\authorinfo{Neil Mitchell\titlenote{The first author is supported by an EPSRC PhD studentship}}
           {University of York, UK}
           {\url{http://www.cs.york.ac.uk/~ndm/}}
\authorinfo{Colin Runciman}
           {University of York, UK}
           {\url{http://www.cs.york.ac.uk/~colin/}}

\maketitle

\begin{abstract}
We describe a transformation which takes a higher-order program, and a produces an equivalent first-order program. Unlike Reynolds style defunctionalisation, it does not introduce any new data types, and the results are more amenable to subsequent analysis operations. Our transformation is implemented, and works on a Core language to which Haskell programs can be reduced. Our method cannot always succeed in removing \textit{all} functional values, but in practice it is remarkably successful. Our transform is useful in combination with analysis operations, such as pattern-match safety or termination checking.
\end{abstract}

\category{D.3}{Software}{Programming Languages}

\terms
languages, transformation

\keywords
Haskell, defunctionalisation, firstification

\section{Introduction}

Higher-order functions are widely used in functional programming languages.  Having functions as first-class values leads to more concise code, but it often complicates analysis methods, such as those for checking pattern-match safety \cite{me:catch} or termination \cite{sereni:higher_order_termination}.

\begin{example}
\label{ex:incList}

Consider this definition of |incList|:

\begin{code}
incList :: [Int] -> [Int]
incList = map (+1)

map :: (alpha -> beta) -> [alpha] -> [beta]
map f []      = []
map f (x:xs)  = f x : map f xs
\end{code}

The definition of |incList| has higher-order features. The function |(+1)| is passed as a functional argument to |map|. The |incList| definition contains a partial application of |map|. The use of first-class functions has led to short code, but we could equally have written:

\begin{code}
incList :: [Int] -> [Int]
incList []      = []
incList (x:xs)  = x+1 : incList xs
\end{code}

Although this first-order variant of |incList| is longer (excluding the library function |map|), it is also more amenable to certain types of analysis. The method presented in this paper transforms the higher-order definition into the first-order one automatically.
\end{example}

The problem we have chosen to tackle is to remove functional values without introducing data values, which can often be used to encode functions -- and thus leaves the complexity of the program the same.

Our defunctionalisation method processes the whole program to remove functional values, without changing the semantics of the program. This idea is not new. As far back as 1972 Reynolds gave a solution, now known as \textit{Reynolds style defunctionalisation} \cite{reynolds:defunc}. Unfortunately, this method effectively introduces a mini-interpreter, which causes problems for analysis tools. Our method produces a program closer to what a human might have written, if denied the use of functional values.

Our method has been implemented in Haskell \cite{haskell}, and operates over the Core language from the York Haskell Compiler \cite{me:yhc_core}. We have used our transformation within the Catch analysis tool \cite{me:catch}, which checks for potential pattern-match errors in Haskell. Catch is a first-order analysis, and without a defunctionalisation method we wouldn't be able to apply Catch to real programs. We have made our defunctionalisation method available as a library on Hackage\footnote{\url{http://hackage.haskell.org/}, under ``firstify''}.

\subsection{Contributions}

Our paper makes the following contributions:

\begin{itemize}
\item We define a defunctionalisation method which, unlike some previous work, does not introduce new data types (\S\ref{sec:overview}, \S\ref{sec:detailed}).
\item Our method can deal with the complexities of a language like Haskell, including type classes, programs using continuation-passing style and monads.
\item Our method makes use of standard transformation steps, but combined in a novel way.
\item We classify where higher-order elements may remain in a resultant program (\S\ref{sec:completeness}).
\item We identify restrictions which guarantee termination, but are not overly limiting (\S\ref{sec:termination}).
\item We have implemented our method, and present measured results for much of the nofib benchmark suite (\S\ref{sec:results}).
\end{itemize}

\noindent There are a number of limitations to our approach, most importantly:

\begin{itemize}
\item Our algorithm is not complete -- it \textit{does not} always succeed in removing all functional values. However, in practice, it is remarkably successful.
\item The transformation can reduce sharing, causing the resulting program to be less efficient and duplicate an arbitrary amount of work. For certain types of analysis the duplication of work is not a problem, for other uses it is a severe problem.
\end{itemize}

\section{Core Language}
\label{sec:core}

\begin{figure}
\ignore\begin{code}
expr  :=  \v -> x           {-"\text{\hspace{1cm}lambda abstraction}"-}
      |   f xs_             {-"\text{\hspace{1cm}function application}"-}
      |   c xs_             {-"\text{\hspace{1cm}constructor application}"-}
      |   x xs_             {-"\text{\hspace{1cm}general application}"-}
      |   v                 {-"\text{\hspace{1cm}variable}"-}
      |   let v = x in y    {-"\text{\hspace{1cm}non-recursive let expression}"-}
      |   case x of alts_   {-"\text{\hspace{1cm}case expression}"-}

alt   := c -> vs_ x         {-"\text{\hspace{1cm}case alternative}"-}
\end{code}
\begin{code}
arityExpr (ELam v x) = 1 + arityExpr x
arityExpr _ = 0
\end{code}

We let |v| range over locally defined variables, |x| and |y| over expressions, |f| over function names and |c| over constructors.
\caption{Core Language.}
\label{fig:core}
\end{figure}

The expression type of our Core language is given in Figure \ref{fig:core}. A program is a mapping of function names to expressions, with a root function named |main|. The arity of a function is the result of applying |arityExpr| to its associated expression. We initially assume there are no primitive functions in our language, but explain how to extend our method to deal with them in \S\ref{sec:primitives}. We allow full Haskell 98 data types, and assume that all Core programs are type correct.

The variable, case, application and lambda expressions are much as they would be in any Core language. We restrict ourselves to non-recursive let expression. (Any recursive let expressions can be removed, with a possible increase in runtime complexity, using the methods described in \cite{me:thesis}.) The constructor expression consists of a constructor and a list of expressions, exactly matching the arity of the constructor. (Any partially applied constructor can be represented as a lambda expression.) A function application consists of a function name and a possibly empty list of argument expressions. If a function is given fewer arguments than its arity we refer to it as \textit{partially-applied}, matching the arity is \textit{fully-applied}, and more than the arity is \textit{over-applied}. We use the meta functions |arity f| and |body f| to denote the arity and body of function |f|. We use the function |rhs| to extract the expression on the right of a case alternative.

Informally, if a program creates functional values at runtime it is higher-order, otherwise it is first-order. Functional values can only be created in two ways: (1) a lambda expression that does not contribute to the arity of its enclosing function; or (2) a partially-applied function application. We therefore make the following definition:

\begin{definition}
A program is \textit{higher-order} if it contains expressions which create or use functional values. An expression creates a functional value if it is a partially-applied function or a lambda expression which does not contribute to the arity of a top-level function. An expression uses a functional value if it is an over-applied functions or a general application.
\end{definition}

\begin{examplerevisit}{\ref{ex:incList}}
The original definition of |incList| is higher-order because it creates functional values with the partial applications of both |map| and |(+)|. The original definition of |map| is higher-order because it uses functional values within a general application. In the defunctionalised version, the program is first-order.
\end{examplerevisit}

\section{Our First-Order Reduction Method}
\label{sec:overview}

Our method works by applying a set of rules non-deterministically until no further rules apply. The rules are grouped in to three categories:

\begin{description}
\item[Simplification:] Many local simplification rules are used, most of which may be found in any optimising compiler \cite{spj:transformation}.
\item[Inlining:] Inlining is a standard technique in optimising compilers \cite{spj:inlining}, and has been studied in depth. Inlining involves replacing an application of a function with the body of the function.
\item[Specialisation:] Specialisation is another standard technique, used to remove type classes \cite{jones:dictionary_free} and more recently to specialise functions to a given constructor \cite{spj:specconstr}. Specialisation involves generating a new function specialised with information about the functions arguments.
\end{description}

Each transformation has the possibility of removing some functional values, but the key contribution of this paper is \textit{how they can be used together} -- including which restrictions are necessary.

We proceed by first giving a brief flavour of how these transformations may be used in isolation to remove functional values. We then discuss the transformations in detail in \S\ref{sec:detailed}.

\subsection{Simplification}

The simplification stage has two purposes: to remove some simple functional values, and to ensure a \textit{normal form} so other rules can apply. The simplification rules are simple, and many are found in optimising compilers. All the rules are given in \S\ref{sec:simplification_detail}.

\begin{example}
\begin{code}
one = (\x -> x) 1
\end{code}

\noindent The simplification rule (lam-app) transforms this function to:

\begin{code}
one = let x = 1 in x
\end{code}\codeexample
\end{example}\smallskip

\noindent Other rules do not eliminate lambda expressions, but put them into a form that other stages can remove.

\begin{example}
\begin{code}
even =  let  one = 1
        in   \x -> not (odd x)
\end{code}

\noindent The simplification rule (let-lam) lifts the lambda outside of the let expression.

\begin{code}
even = \x ->  let  one = 1
              in   not (odd x)
\end{code}

\noindent In general this transformation may cause duplicate computation to be performed, an issue we return to in \S\ref{sec:sharing}.
\end{example}


\subsection{Inlining}

We use inlining to remove functions which return data constructors containing functional values. A frequent source of data constructors containing functional values is the dictionary implementation of type classes \cite{wadler:type_classes}.

\begin{example}
\begin{code}
main = case  eqInt of
             (a,b) -> a 1 2

eqInt = (primEqInt, primNeqInt)
\end{code}

Both components of the |eqInt| tuple, |primEqInt| and |primNeqInt|, are functional values. We can start to remove these functional values by inlining |eqInt|:

\begin{code}
main = case  (primEqInt, primNeqInt) of
             (a,b) -> a 1 2
\end{code}

\noindent The simplification stage can now turn the program into a first-order variant, using rule (case-con) from \S\ref{sec:simplification_detail}.

\begin{code}
main = primEqInt 1 2
\end{code}\codeexample
\end{example}

\subsection{Specialisation}

We use specialisation to remove lambda expressions that given as arguments to functions. Specialisation works by replacing a function application with a variant with some information from the arguments frozen in. In effect, the lambda expressions are passed at transformation time.

\begin{example}
\begin{code}
notList xs = map not xs
\end{code}

\noindent Here the |map| function takes the functional value |not| as its first argument. We can create a variant of |map| specialised to this argument:

\begin{code}
map_not x = case  x of
                  []    -> []
                  y:ys  -> not y : map_not ys

notList xs = map_not xs
\end{code}

\noindent The recursive call in |map| is replaced by a recursive call to the specialised variant. We have eliminated all functional values.
\end{example}

\subsection{Goals}
\label{sec:goals}

We define a number of goals: some are \textit{essential}, and others are \textit{desirable}. If essential goals make desirable goals unachievable in full, we still aim to do the best we can.

\subsubsection*{Essential}

\paragraph{Preserve the result computed by the program.} By making use of established transformations, total correctness is relatively easy to show.

\paragraph{Ensure the transformation terminates.} The issue of termination is much harder. Both inlining and specialisation could be applied in ways that diverge. In \S\ref{sec:termination} we develop a set of criteria to ensure termination.

\paragraph{Recover the original program.} Our transformation is designed to be performed before analysis. It is important that the results of the analysis can be presented in terms of the original program. We need a method for transforming expressions in the resultant program into equivalent expressions in the original program.

\paragraph{Introduce no data types.} Reynolds method introduces a new data type that serves as a representation of functions, then embeds an interpreter for this data type into the program. We aim to eliminate the higher-order aspects of a program \textit{without} introducing any new data types. By not introducing any data types we avoid introducing an interpreter, which is often a bottleneck for subsequent analysis. By composing our transformation out of existing transformations, none of which introduces data types, we can easily ensure that our resultant transformation does not introduce data types.


\subsubsection*{Desirable}

\paragraph{Remove all functional values.} We aim to remove as many functional values as possible. In \S\ref{sec:completeness} we make precise where functional values may appear in the resultant programs. If a totally first-order program is required, Reynolds' method can always be applied after our transformation. Applying our method first will cause Reynolds' method to introduce fewer additional data types and generate a smaller interpreter.

\paragraph{Preserve the space/sharing behaviour of the program.} In the expression |let y = f x in y + y|, according to the rules of lazy evaluation, |f x| will be evaluated at most once. It is possible to inline the let binding to give |f x + f x|, but this expression evaluates |f x| twice. This transformation is valid in Haskell due to referential transparency, and will preserve both semantics and termination, but may increase the amount of work performed. In an impure or strict language, such as ML \cite{ml}, this transformation may change the semantics of the program.

Our goals are primarily for analysis of the resultant code, not to compile and execute the result. Because we are not interested in performance, we permit the loss of sharing in computations if to do so will remove functional values. However, we will avoid the loss of sharing where possible, so the program remains closer to the original.

\paragraph{Minimize the size of the program.} Previous defunctionalisation methods have reflected a concern to avoid undue code-size increase \cite{chin:higher_order_removal}. A smaller resultant program would be desirable, but not at the cost of clarity.

\paragraph{Make the transformation fast.} The implementation must be sufficiently fast to permit proper evaluation. Ideally, when combined with a subsequent analysis phase, the defunctionalisation should not take an excessive proportion of the runtime.


\section{Method in Detail}
\label{sec:detailed}

\begin{comment}
% paper haskell
\begin{code}
data Prog = Prog deriving Eq
simplify,arity,inline,specialise :: Prog -> Prog
\end{code}
\end{comment}

This section covers the rules that are applied non-deterministically. Many programs require a combination or rules to be applied, for example, the initial |incList| example requires rules from both simplification and specialisation.

We have implemented our steps in a monadic framework to deal with issues such as obtaining unique free variables and tracking termination constraints. But to simplify the presentation here, we ignore these issues -- they are mostly tedious engineering concerns, and do not effect the underlying algorithm.

\subsection{Simplification}
\label{sec:simplification_detail}

\begin{figure}
\begin{simplify}

\simp{app-app}{
\ignore\begin{code}
(x xs_) ys_
    => x xs_ ys_
\end{code}}

\simp{fun-app}{
\ignore\begin{code}
(f xs_) ys_
    => f xs_ ys_
\end{code}}

\simp{lam-app}{
\ignore\begin{code}
(\v -> x) y
    => let v = y in x
\end{code}}

\simp{let-app}{
\ignore\begin{code}
(let v = x in y) z
    => let v = x in y z
\end{code}}

\simp{case-app}{
\ignore\begin{code}
(case x of {p_1 -> y_1 ; ... ; p_n -> y_n}) z
    => case x of {p_1 -> y_1 z ; ... ; p_n -> y_n z}
\end{code}}

\simp{case-con}{
\ignore\begin{code}
case c xs_ of {... ; c vs_ -> y ; ...}
    => let vs_ = xs_ in y
\end{code}}

\simp{case-let}{
\ignore\begin{code}
case (let v = x in y) of alts_
    => let v = x in (case y of alts_)
\end{code}}

\simp{case-case}{
\ignore\begin{code}
case (case x of {... ; c vs_ -> y ; ...}) of alts_
    => case x of {... ; c vs_ -> case y of alts_ ; ...}
\end{code}}

\simp{case-lam}{
\ignore\begin{code}
case x of {... ; c vs_ -> \v -> y ; ...}
    => \z -> case  x of
                   {... z ; c vs_ -> (\v -> y) z ; ... z}
\end{code}}

\simp{eta}{
\ignore\begin{code}
f xs_
    => \v -> f xs_ v
    where arity f > length xs_
\end{code}}

\end{simplify}
\caption{Standard Core simplification rules.}
\label{fig:simplify}
\end{figure}


\begin{figure}
\begin{simplify}

\simp{bind-lam}{
\ignore\begin{code}
let v = (\w -> x) in y
    => y[v / \w -> x]
\end{code}}

\simp{bind-box}{
\ignore\begin{code}
let v = x in y
    => y[v / x]
    where x {-" \text{ is a boxed lambda (see \S\ref{sec:inlining})} "-}
\end{code}}

\simp{let-lam}{
\ignore\begin{code}
let v = x in \w -> y
    => \w -> let v = x in y
\end{code}}

\end{simplify}
\caption{Lambda Simplification rules.}
\label{fig:lambda_simplify}
\end{figure}

The simplification rules have the goal of moving lambda expressions upwards, and introducing lambdas for partially applied functions. The rules comprise standard simplification rules given in Figure \ref{fig:simplify}, which are found in most optimising compilers, such as GHC \cite{spj:transformation}. We also make use of additional rules which deal specifically with lambda expressions, given in Figure \ref{fig:lambda_simplify}. All of the simplification rules are correct individually. The rules are applied to any subexpression, as long as any rule matches. We believe that the combination or rules from Figure \ref{fig:lambda_simplify} and Figure \ref{fig:simplify} are confluent.

\subsubsection{Lambda Introduction}

The (eta) rule inserts lambdas in preference to partial applications, using $\eta$-expansion. For each partially applied function, a lambda expression is inserted to ensure that the function is given at least as many arguments as its associated arity.

\begin{example}
\ignore\begin{code}
(.) f g x = f (g x)

even = (.) not odd
\end{code}

\noindent Here the functions |(.)|, |not| and |odd| are all unsaturated. Lambda expressions can be inserted to saturate these applications.

\begin{code}
even = \x -> (.) (\y -> not y) (\z -> odd z) x
\end{code}

\noindent Here the |even| function, which previously had three instances of partial application, has three lambda expressions inserted. Now each function is fully-applied. This replaces partial application with lambda expressions, and has the advantage of making functional values more explicit, permitting other transformations.
\end{example}

\subsubsection{Lambda Movement}
\label{sec:sharing}

The (bind-lam) rule inlines a lambda bound in a let expression. The (let-lam) rule can be responsible for a reduction in sharing:

\begin{example}
\begin{code}
f x =  let  i = expensive x
       in   \j -> i + j

main xs = map (f 1) xs
\end{code}

Here |(expensive 1)| is computed once and saved. Every application of the functional argument within |map| performs a single |(+)| operation. After applying the (let-lam) rule we get:

\begin{code}
f x =  \j ->  let  i = expensive x
              in   i + j
\end{code}

Now |expensive| is recomputed for every element in |xs|. We include this rule in our simplifier, focusing on functional value removal at the expense of sharing.
\end{example}

\subsection{Inlining}
\label{sec:inlining}

We use inlining of top-level functions as the first stage in the removal of functional values stored within a data value -- for example |Just (\x -> x)|. We refer to expressions that evaluate to functional values inside data values as \textit{boxed lambdas}. If a boxed lambda is bound in a let expression, we substitute the let binding, using the (bind-box) rule from Figure \ref{fig:lambda_simplify}. We only inline a function if two conditions both hold: (1) the function's body is a boxed lambda; (2) the function application occurs within a case scrutinee.

\begin{figure}
\begin{code}
isBox (ECon c xs_     )  =  any isLambda xs_ ||  any isBox xs_
isBox (ELet v x y     )  =  isBox y
isBox (ECase x alts_  )  =  any (isBox . rhs) alts_
isBox (EFun f xs_     )  =  isBox (fromLambda (body f))
isBox _                  =  False

fromLambda (ELam v x)  = fromLambda x
fromLambda x           = x

isLambda (ELam v x)  = True
isLambda _           = False
\end{code}

The |isBox| function as presented may not terminate, but by simply keeping a list of followed functions, we can assume the result is |False| in any duplicate call. This modification does not change the result of any previously terminating evaluations.
\caption{The |isBox| function, to test if an expression is a boxed lambda.}
\label{fig:boxed_lambda}
\end{figure}

\begin{definition}
An expression |e| is a boxed lambda if |isBox e == True|, where |isBox| is defined as in Figure \ref{fig:boxed_lambda}.
\end{definition}

\begin{example}
Recalling that |[e]| is shorthand for |(:) e []|, where |(:)| is the cons constructor, the following expressions are boxed lambdas:

\ignore\begin{code}
[\x -> x]
(Just [\x -> x])
(let y = 1 in [\x -> x])
[Nothing, Just (\x -> x)]
\end{code}

\noindent The following are \textit{not} boxed lambdas:

\ignore\begin{code}
\x -> id x
[id (\x -> x)]
id [\x -> x]
\end{code}

The final expression \textit{evaluates to} a boxed lambda, but this information is hidden by the |id| function. We rely on specialisation to remove any boxed lambdas passed to functions.
\end{example}

\begin{definition}
The inlining transformation is specified by:

\ignore\begin{code}
case (f xs_) of alts_
    => case (y xs_) of alts_
    where
        y = body f
        {-" \text{If } "-} isBox (f xs_) {-" \text{ evaluates to } "-} True
\end{code}\codeexample
\end{definition}\bigskip

As with the simplification stage, there may be some loss of sharing if the definition being inlined has arity 0 -- a constant applicative form (CAF). A Haskell implementation computes these expressions only once, and reuses their value as necessary. If they are inlined, this sharing will be lost.

\subsection{Specialisation}

For each application of a top-level function in which at least on argument has a subexpression which is a lambda, a specialised variant is created, and used where applicable. The process follows the same pattern as constructor specialisation \cite{spj:specconstr}, but applies where function arguments are lambda expressions, rather than known constructors. Examples of common functions whose applications can usually be made first-order by specialisation include |map|, |filter|, |foldr| and |foldl|.

The specialisation transformation makes use of \textit{templates}. A template is an expression where some sub-expressions are omitted, denoted by the |hole| symbol. The process of specialisation proceeds as follows:

\begin{enumerate}
\item Find all function applications which need specialising, and generate templates (see \S\ref{sec:need_templates}).
\item Abstract templates, replacing some components with |hole| (see \S\ref{sec:abstract_templates}).
\item For each template, generate a function specialised to that template (see \S\ref{sec:generate_functions}).
\item For each subexpression matching a template, replace it with the generated function (see \S\ref{sec:use_templates}).
\end{enumerate}

\begin{example}
\label{ex:map_id}
\begin{code}
main xs = map (\x -> x) xs

map f xs = case  xs of
                 []    -> []
                 y:ys  -> f y : map f ys
\end{code}

Specialisation first finds the application of |map| in |main|, and generates the template \ignore|map (\x -> x) hole|. It then generates a unique name for the template (we choose |map_id|), and generates an appropriate function body. Next all calls matching the template are replaced with calls to |map_id|, including the call to |map| within the freshly generated |map_id|.

\begin{code}
main xs = map_id xs

map_id xs = case  xs of
                  []    -> []
                  y:ys  -> y : map_id ys
\end{code}

\noindent The resulting code is first-order.
\end{example}

\begin{comment}
\begin{figure}

\begin{code}
shouldSpecialise :: Expr -> Bool
shouldSpecialise (EFun f xs_) = any (\x -> isLambda x || isBox x) (universe (EFun f xs_))
shouldSpecialise _ = False

abstractTemplate :: Expr -> Expr
abstractTemplate x =
    transform (\x -> if abstract x then hole else x) $
    abstractVars (freeVars x) x

abstract :: Expr -> Bool
abstract (ECon c xs_) = all abstract xs_
abstract (EFun f xs_) = all abstract xs_
abstract (EApp x xs_) = all abstract xs_
abstract (ELet v x y) = abstract x && abstract (abstractVars [v] y)
abstract (ECase x alts_) = abstract x && all alt alts_
   where alt (EAlt c vs_ x) = abstract (abstractVars vs_ x)
abstract x = (x == hole)

abstractVars :: [VarName] -> Expr -> Expr
abstractVars vs x = ellipses
    -- replace the variables |vs| in |x| with |abstract|
    -- respecting variables rebound locally
\end{code}
\caption{Template generation function.}
\label{fig:template_generation}
\end{figure}
\end{comment}

\subsubsection{Generating Templates}
\label{sec:need_templates}

The idea is to generate templates for all function calls which pass functional values. Given a expression |x| a template is generated if: (1) |x| is a function application; and (2) at least one of the sub-expressions of |x| is either a lambda or a boxed lambda (see \S\ref{sec:inlining}). In all cases, the template generated is simply |x|.

\begin{example}
The following expressions generate templates:

\h{exprlist}\begin{code}
id (\x -> x)
map f [\x -> x]
id (Just (\x -> x + 1))
f (\v -> v) True
\end{code}
\end{example}


\subsubsection{Abstracting Templates}
\label{sec:abstract_templates}

We perform abstraction to reduce the number of different templates required, by replacing non-functional expressions with |hole|. For each sub-expression |x| in a template, it can be replaced with |hole| if the following two conditions hold:

\begin{enumerate}
\item |e| is not, and does not contain, any expressions which are either lambda expressions or boxed lambdas, e.g. we cannot replace |(\x -> x)| or |(let y = \x -> x in y)| with |hole|.
\item None of the free variables in |e| are bound in the template, e.g. we cannot replace the expression |f v| in |let v = 1 in f v| with |hole|, as the variable |v| is bound within the template.
\end{enumerate}

\begin{example}
\noindent\begin{tabular}{@@{}ll}
Template & Abstract Template \\
|id (\x -> x)|                & |id (\x -> x)| \\
|id (Just (\x -> x))|         & |id (Just (\x -> x))| \\
|id (\x -> let y = 12 in 4)|  & |id (\x -> hole)| \\
|id (\x -> let y = 12 in x)|  & |id (\x -> let y = hole in x)| \\
\end{tabular}
\smallskip

In all these examples, the |id| function has an argument which has a lambda expression as a subexpression. In the final two cases, there are subexpressions which do not depend on variables bound within the lambda -- these have been removed and replaced with |hole|.
\end{example}

\subsubsection{Generating Functions}
\label{sec:generate_functions}

Given a template, to generate an associated function, a unique function name is allocated to the template. For each occurrence of |hole| in a template a fresh argument variable is assigned. The body is produced by unfolding the outer function symbol in the template once.

\begin{examplerevisit}{\ref{ex:map_id}}
Consider the template |map (\x -> x) hole|. Let |v_1| be the fresh argument variable for the single |hole| placeholder, and |map_id| be the function name:

\begin{code}
map_id = \v_1 -> map (\x -> x) v_1
\end{code}

\noindent We unfold the definition of |map| once:

\begin{code}
map_id = \v_1 ->  (\f xs -> case  xs of
                                  []    -> []
                                  y:ys  -> f y : map f ys)
                  (\x -> x) v_1
\end{code}

\noindent After the simplification rules from Figure \ref{fig:lambda_simplify}, we obtain:

\begin{code}
map_id = \v_1 ->  let  xs = v_1
                  in   case  xs of
                             []    -> []
                             y:ys  -> y : map (\x -> x) ys
\end{code}\codeexample
\end{examplerevisit}

\subsubsection{Using Templates}
\label{sec:use_templates}

After a function has been generated for each template, every expression matching a template can be replaced by a call to the new function. Every subexpression corresponding to |hole| is passed as an argument.

\begin{exampleany}{\ref{ex:map_id} (continued)}
\begin{code}
map_id = \v_1 ->  let  xs = v_1
                  in   case  xs of
                             []    -> []
                             y:ys  -> y : map_id ys
\end{code}

\noindent We now have a first-order definition.
\end{exampleany}

\subsection{Confluence}

The transformations we have presented are not confluent. Consider the expression |id ((\x -> x) 1)|. We can either apply specialisation, or the (lam-app) rule. The first will involve the creation of an additional function definition, while the second will not.

We conjecture that the rules in each of the separate categories are confluent. In order to ensure a deterministic application of the rules we always favour rules first from the simplification stage, then the inlining stage, and finally the specialisation stage. Doing so will in some cases lead to a superior result being produced. \todo{superior result?}

\subsection{Primitive Functions}
\label{sec:primitives}

Primitive functions do not have an associated body, and therefore cannot be examined or inlined. We make just two simple changes to support primitives.

\begin{enumerate}
\item We define that a primitive application is \textit{not} a boxed lambda.
\item We restrict specialisation so that if a function to be specialised is actually a primitive, no template is generated. The reason for this restriction is that the generation of code associated with a template requires a one-step unfolding of the function, something which cannot be done for a primitive.
\end{enumerate}

These restrictions mean that some programs using primitive functions cannot be made first-order.

\begin{example}
\begin{code}
main = seq (\x -> x) 42
\end{code}

Here a functional value is passed as the first argument to the primitive |seq|. As we are not able to peer inside the primitive, and must preserve its interface, we cannot remove this functional value. For most primitives, such as arithmetic operations, the types ensure that no functional values are passed as arguments. However, the |seq| primitive is of type \ignore|alpha -> beta -> beta|, allowing any type to be passed as either of the arguments, including functional values.

Some primitives not only \textit{permit} functional values, but actually \textit{require} them. For example, the |primCatch| function within the Yhc standard libraries implements the Haskell exception handling function |catch|. The type of |primCatch| is \ignore|alpha -> (IOError -> alpha) -> alpha|, taking an exception handler as one of the arguments.
\end{example}

\subsection{Recovering Input Expressions}

Specialisation is the only stage which introduces new function names. In order to translate an expression in the result program to an equivalent expression in the input program, it is sufficient to replace all generated function names with their associated template, supplying all the necessary variables.

\section{Examples}
\label{sec:examples}

We now give two examples. Our method can convert the first example to a first-order equivalent, but not the second.

\begin{examplename}{Inlining Boxed Lambdas}
\label{ex:inlining_boxed_lambdas}
An earlier version of our defunctionaliser inlined boxed lambdas everywhere they occurred. Inlining boxed lambdas means the |isBox| function does not have to examine the body of applied functions, and is therefore simpler. However, it was unable to cope with programs like this one:

\begin{code}
main = map ($ 1) gen
gen = (\x -> x) : gen
\end{code}

The |gen| function is both a boxed lambda and recursive. If we inlined |gen| initially the method would not be able to remove all lambda expressions. By first specialising |map| with respect to |gen|, and waiting until |gen| is the subject of a case, we are able to remove the functional values. This operation is effectively deforestation \cite{wadler:deforestation}, which also only performs inlining within the subject of a case.
\end{examplename}

\begin{examplename}{Functional Lists}
\label{ex:functional_lists}
Sometimes lambda expressions are used to build up lists which can have elements concatenated onto the end. Using Hughes lists \cite{hughes:lists}, we can define:

\begin{code}
nil = id
snoc x xs = \ys -> xs (x:ys)
list xs = xs []
\end{code}

This list representation provides |nil| as the empty list, but instead of providing a |(:)| or ``cons'' operation, it provides |snoc| which adds a single element on to the end of the list. The function |list| is provided to create a standard list. We are unable to defunctionalise such a construction, as it stores unbounded information within closures. We have seen such constructions in both the |lines| function of the HsColour program, and the |sort| function of Yhc. However, there is an alternative implementation of these functions:

\begin{code}
nil = []
snoc = (:)
list = reverse
\end{code}

We have benchmarked these operations in a variety of settings and the list based version appears to use approximately 75\% of the memory, and 65\% of the time required by the function-based solution. We suspect there are sometimes efficiency savings to storing data in closures, but not for snoc-based lists.
\end{examplename}


\section{Restricted Completeness}
\label{sec:completeness}

Our method would be \textit{complete} if it made all programs to first-order. In this section we give some results about where expressions creating functional values can remain in a resulting program.

\subsection{Proposition}

After transformation, there will be no instances of partial application, and all lambda expressions will be unreachable (never be evaluated at runtime), provided:

\begin{enumerate}
\item The termination criteria do not curtail defunctionalisation (see \S\ref{sec:termination}).
\item No primitive function receives a functional value, or returns a functional value.
\item The main function has a type that ensures it neither receives a functional argument, nor returns a functional result.
\end{enumerate}

We prove this proposition with a series of lemmas about the resultant program.

\subsection{Lemmas}

\begin{lemma}{No partial application}

The (eta) rule removes partial application, and at the end of transformation, no further rules apply -- therefore there can be no partial application in the resultant program.
\end{lemma}

\begin{lemma}{The first argument of a general application must be a local variable}

The rules (app-app), (fun-app), (lam-app), (let-app) and (case-app) mean the first argument to a general application must be a variable of a constructor application. All constructor applications are fully applied, and therefore cannot return a function, so type safety ensures they cannot be the first argument of an application. Therefore, the first argument of an application is a local variable.
\end{lemma}

\begin{lemma}{The direct parent of a lambda expression must be one of: the root of a function definition; an argument to an application; a lambda expression; an argument to a constructor}

A lambda cannot be the scrutinee in a case expression as it would not be well typed. A lambda cannot be an argument to a function as it would be removed by specialisation. All other possible lambda positions are removed by the rules (lam-app), (case-lam), (bind-lam) and (let-lam).
\end{lemma}

\begin{lemma}{The outer parent of a boxed lambda must be one of: the root of a function definition; an argument to an application; a lambda expression}

For the outer parent of a boxed lambda we refer to an expression which is not itself a boxed lambda, but contains a boxed lambda as an immediate subexpression. Using the definition of |isBox| from Figure \ref{fig:boxed_lambda} we restrict the outer parent of a boxed lambda to the binding of a let, the scrutinee of a case, and argument to a function or one of the locations mentioned in the lemma. We remove the binding of a let with (bind-box) and the argument to a function with specialise.

To remove a boxed lambda from a case scrutinee we observe that a boxed lambda must be a constructor application, a let expression, a case expression or a function application. The first three are removed with the rules (case-con), (case-let) and (case-case). The function application based boxed lambda is removed with inlining.
\end{lemma}

\begin{lemma}{A boxed lambda must have a type that may contain a functional value}

The base case of a boxed lambda is a constructor application to a lambda, which trivially contains a lambda. For let and case, the type of the expression is the type of the inner boxed lambda. The remaining case is if |((\vs_ -> b) xs_)| is able to contain a functional value. As |b| must be a boxed lambda, i.e. a constructor wrapping a lambda, any application and abstraction operations alone cannot examine the constructor, so cannot remove the functional type.
\end{lemma}

\begin{lemma}{A function whose root is a boxed lambda must be called from the argument of an application or inside a lambda}

An application of a function whose root is a boxed lambda is itself a boxed lambda, therefore the restrictions on where a boxed lambda can reside apply to function applications whose root is a boxed lambda.
\end{lemma}

\begin{lemma}{All lambda expressions are unreachable}

The |main| function cannot be a boxed lambda, as that would be a functional value, and is disallowed by restrictions on |main|. There remain only two possible locations for lambda expressions or boxed lambdas:

\h{exprlist}\begin{code}
v (\w -> ellipses)
\v -> (\w -> ellipses)
\end{code}

Neither of these constructs binds a lambda expression to a value, therefore in the first case |v| cannot be bound to a lambda. If |v| is not a lambda, then type checking means that |v| \textit{must} be the value $\bot{}$, and therefore the lambda will never be evaluated. In the second case, the lambda must be contained with a lambda, which itself must be either in a lambda or a general application -- and again will not be evaluated.
\end{lemma}

\begin{lemma}{There is no partial application and all lambda expressions are unreachable}

By combining the lemmas that there is no partial application and that all lambda expressions are unreachable.
\end{lemma}

It is instructive to note that during the proof every rule has been used, and that the removal of any single rule would invalidate the proof. While this does not prove that each step is necessary, it does show where the motivation for each rule comes from.

\subsection{Residual Higher-Order Programs}
\label{sec:example_residual}

The following programs all remain higher-order after applying our method, although none will actually create higher-order values at runtime.

\begin{example}
\begin{code}
main = bottom (\x -> x)
\end{code}

We use the expression |bottom| to indicate a computation that evaluates to $\bot{}$ -- either a call to |error| or a non-termination. This program will evaluate to $\bot{}$, and will never evaluate the lambda expression.
\end{example}

\begin{example}
\begin{code}
nothing = Nothing
main = case  nothing of
             Nothing  -> 1
             Just f   -> f (\x -> x)
\end{code}

In this example the lambda expression is never reached because that particular branch of the case expression is never taken.
\end{example}

\subsection{Consequences}

Our proposition means that replacing all lambda expressions in the resultant program with $\bot{}$ will give an identical program. Another way of viewing this proposition is that after transformation the program will be first-order at runtime, even if there are some higher-order features present in the source program. In addition, any uses of functional values are guaranteed to actually be operating on $\bot{}$, as no functional values could have been created. Therefore, given the above restrictions, the following rewrites are valid:

\ignore\begin{code}
(\v -> x)  => bot    {-"\hspace{3mm} \hbox{if not contributing to the arity of a function}"-}
x xs_      => bot
f xs_      => bot    {-"\hspace{3mm} \hbox{if } "-} arity f /= length xs_
\end{code}

After performing these rewrites, the program is guaranteed to be first-order.

\section{Proof of Termination}
\label{sec:termination}

\begin{comment}
We can remove all data types by encoding them as functions, as described in \citet{naylor:reduceron}. If we then had a transformation which made the program first-order \textit{without} introducing any data types, we would end up with a program without data or closures, which is incapable of storing an unbounded amount of information. Since with higher-order functions we can implement a Turing machine \cite{turing:halting}, and without an unbounded store we cannot, such a transformation cannot exist.
\end{comment}

Our algorithm, as it stands, may not terminate. In order to ensure termination, it is necessary to bound both the inlining and specialisation stages. In this section we develop a mechanism to ensure termination, by first looking at how non-termination may arise.

\subsection{Termination of Simplification}
\label{sec:termination_simplification}

\begin{figure}
\textsf{\noindent\begin{tabular}{@@{}l@@{}l}
[x,y,z]                                          \\
app(lam(x),y)    & |->| let(y,x)                     \\
app(case(x,y),z) & |->| case(x,app(y,z))             \\
app(let(x,y),z)  & |->| let(x,app(y,z))              \\
case(let(x,y),z) & |->| let(x,case(y,z))             \\
case(con(x),y)   & |->| let(x,y)                     \\
case(x,lam(y))   & |->| lam(case(x,app(lam(y),var))) \\
let(lam(x),y)    & |->| lam(let(x,y))                \\
\end{tabular}}
\caption{Encoding of termination simplification.}
\label{fig:term_simplification}
\end{figure}

In order to check the termination of the simplifier we have used the AProVE system \cite{aprove} to model our rules as a \textit{term rewriting system}, and check its termination. An encoding of a simplified version of the rules from Figures \ref{fig:simplify} and \ref{fig:lambda_simplify} is given in Figure \ref{fig:term_simplification}. We have encoded rules by considering what type of expression is transformed by a rule. For example, the rule replacing |(\v -> x) y| with |let v = y in x| is expressed as a rewrite replacing \textsf{app(lam(x),y)} with \textsf{let(y,x)}. The names of binding variables with expressions have been ignored. To simplify the encoding, we have only considered applications with one argument. The rules are applied non-deterministically at any suitable location, so faithfully model the behaviour of the original rules.

The encoding of the (bind-box) and (bind-lam) rules is excluded. Given these rules, there are non terminating sequences. For example:

\ignore\begin{code}
(\x -> x x) (\x -> x x)
   => -- (lam-app) rule
let x = \x -> x x in x x
   => -- (bind-lam) rule
(\x -> x x) (\x -> x x)
\end{code}

Such expressions are a problem for GHC, and can cause the compiler to non-terminate if encoded as data structures \cite{spj:inlining}. Other transformation systems \cite{chin:higher_order_removal} make use of type annotations to ensure these reductions terminate. To guarantee termination, we apply (bind-lam) or (bind-box) at most $n$ times in any definition body. If the body is altered by either inlining or specialisation, we reset the count. Currently we have set $n$ to 1000, and have never had this limit reached. This limited is intended to give a strong guarantee of termination, and will only be necessary rarely -- hence the high bound.

\subsection{Termination of Inlining}

A standard technique to ensure termination of inlining is to refuse to inline recursive functions \cite{spj:inlining}. For our purposes, this non-recursive restriction is too cautious as it would leave residual lambda expressions in cases such as Example \ref{ex:inlining_boxed_lambdas}. We first present a program which causes our method to fail to terminate, then our means of ensuring termination.

\begin{example}
\begin{code}
data B alpha = B alpha
f = case  f of
          B _ -> B (\x -> x)
\end{code}

The |f| inside the case is a candidate for inlining:

\ignore\begin{code}
case f of B _ -> B (\x -> x)
    => -- inlining rule
case (case f of B _ -> B (\x -> x)) of B _ -> B (\x -> x)
    => -- (case-case) rule
case f of B _ -> case B (\x -> x) of B _ -> B (\x -> x)
    => -- (case-con) rule
case f of B _ -> B (\x -> x)
\end{code}

\noindent So this expression would cause non-termination.
\end{example}

To avoid such problems, we permit inlining a function |f|, at all use sites within the definition of a function |g|, but only once per pair |(f,g)|. In the previous example we would inline |f| within its own body, but only once. Any future attempts to inline |f| within this function would be disallowed, although |f| could still be inlined within other function bodies. This restriction is sufficient to ensure termination of inlining. Given $n$ functions, there can only be $n^2$ possible inlining steps, each for possibly many application sites.


\subsection{Termination of Specialisation}
\label{sec:termination_specialisation}

The specialisation method, left unrestricted, also may not terminate.

\begin{example}
\label{ex:wrap}
\begin{code}
data Wrap alpha = Wrap (Wrap alpha) | Value alpha

f x = f (Wrap x)
main = f (Value head)
\end{code}

In the first iteration, the specialiser generates a version of |f| specialised for the argument |Value head|. In the second iteration it would specialise for |Wrap (Value head)|, then in the third with |Wrap (Wrap (Value head))|. Specialisation would generate an infinite number of specialisations of |f|.
\end{example}

\begin{figure}
\begin{equation*}
\frac{s \unlhd t_i \text{ for some } i}{s \unlhd \sigma(t_1,\ldots,t_n)}
\;\;\;\;\;\;\;\;\;\;\;
\frac{\sigma_1 \equiv \sigma_2,
        s_1 \unlhd t_1, \ldots , s_n \unlhd t_n}
       {\sigma_1 (s_1,\ldots,s_n) \unlhd \sigma_2 (t_1,\ldots,t_n)}
\end{equation*}
\caption{Homeomorphic embedding relation.}
\label{fig:homeomorphic}
\end{figure}

To ensure we only specialise a finite number of times we use a homeomorphic embedding \cite{leuschel:homeomorphic}, given in Figure \ref{fig:homeomorphic}. The homeomorphic embedding $\unlhd$ is a well-quasi order, meaning there are no infinite admissible sequences. A sequence $s_1,s_2 \ldots$ is admissible if there are no $i < j$ such that $s_i \unlhd s_j$. This property only holds provided all elements are expressions over a finite alphabet.

For each function, we associate a set $S$, of expressions. After generating a template $t$, we only specialise with that template if $\forall s \in S \bullet \neg(s \unlhd t)$. If $S$ is already admissible, then the sequence $S$ with $t$ added at the end is still admissible. After specialising with a template we add that template to the set $S$ associated with that expression. When we create a new function based on a template, we copy the $S$ associated with the function in which the specialisation is performed.

One of the conditions for termination of homeomorphic embedding is that there is only a finite alphabet. To ensure this condition, we first consider all variables and literals to be equivalent. However, this is not sufficient. During the process of specialisation we create new functions, and these new functions are new symbols in our language. So we only use function names from the original input program. Every template has a correspondence with an expression in the original program. We perform the homeomorphic embedding test only after transforming all templates into their original equivalent expression.

\begin{examplerevisit}{\ref{ex:wrap}}
Using homeomorphic embedding, we again generate the specialised variant of |f (Value head)|. Next we generate the template |f (Wrap (Value head))|. However, |f (Value head)| $\unlhd{}$ |f (Wrap (Value head))|, so the new template would not be used.
\end{examplerevisit}

Forbidding homeomorphic embeddings in specialisation still allows full defunctionalisation in most simple examples, but there are examples where it terminates prematurely.

\begin{example}
\begin{code}
main y = f (\x -> x) y
f x y = fst (x, f x y) y
\end{code}

Here we first generate a specialised variant of |f (\x -> x) y|.  If we call the specialised variant |f'|, we have:

\begin{code}
f' y = fst (\x -> x, f' y) y
\end{code}

Note that the recursive call to |f| has also been specialised. We now attempt to generate a specialised variant of |fst|, using the template |fst (\x -> x, f' y) y|. Unfortunately, this template is an embedding of the template we used for |f'|, so we do not specialise and the program remains higher-order. But if we did permit a further specialisation, we would obtain the first-order equivalent:

\begin{code}
f' y = fst' y y
fst' y_1 y_2 = y_2
\end{code}\codeexample
\end{example}\smallskip

This example may look slightly obscure, but similar situations occur commonly with the standard translation of dictionaries. Often, classes have default methods, which call other methods in the same class. These recursive class calls often pass dictionaries, embedding the original caller even though no recursion actually happens.

To alleviate this problem, instead of storing one set $S$, we store a sequence of sets, $S_1 \ldots S_n$ -- where $n$ is a small positive number, constant for the duration of the program. Instead of adding to the set $S$, we now add to the lowest set $S_i$ where adding the element will not violate the admissible sequence. Each of the sets $S_i$ is still finite, and there are a finite number ($n$) of them, so termination is maintained.

By default our defunctionalisation program uses 8 sets. In the results table given in \S\ref{sec:results}, we have given the minimum possible value of $n$ to remove all lambda expressions within each program.

\subsection{Termination as a Whole}

Given an initial program, inlining and specialisation rules will only apply a finite number of times. The simplification rules are terminating on their own, so when combined, all the rules will terminate.

\section{Results}
\label{sec:results}

\subsection{Benchmark Tests}

\begin{table}
\caption{Results of defunctionalisation on the nofib suite.}
\label{tab:results}

\smallskip

\textbf{Name} is the name of the program;
\textbf{Bound} is the numeric bound used for termination (see \S\ref{sec:termination_specialisation});
\textbf{HO Create} the number of lambda expressions and under-applied functions, first in the input program and then in the output program;
\textbf{HO Use} the number of application expressions and over-applied functions;
\textbf{Time} the execution time of our method in seconds;
\textbf{Size} the change in the program size measured as the number of lines of Core.

\smallskip\smallskip

\begin{tabular*}{\linewidth}{@@{}l@@{\extracolsep{\fill}}rrrrrrr@@{}}
\textbf{Name} & \textbf{Bound} & \multicolumn{2}{c}{\textbf{HO Create}} & \multicolumn{2}{c}{\textbf{HO Use}} & \textbf{Time} & \textbf{Size} \\
\vspace{-1ex} \\
\multicolumn{8}{@@{}l}{Programs curtailed by a termination bound:} \\
cacheprof	 & 8	 & 611	 & 44	 & 686	 & 40  & 1.8	 & 2\% \\
grep	 & 8	 & 129	 & 9	 & 108	 & 22	 & 0.8  & 40\% \\
lift	 & 8	 & 187	 & 123	 & 175	 & 125  & 1.2	 & -6\% \\
prolog	 & 8	 & 308	 & 301	 & 203	 & 137	 & 1.1  & -5\% \\
\vspace{-1ex} \\
\multicolumn{8}{@@{}l}{All other programs:} \\
ansi	 & 4	 & 239	 & 0	 & 187	 & 2   & 0.5	 & -29\% \\
bernouilli	 & 4	 & 240	 & 0	 & 190	 & 2  & 0.3	 & -32\% \\
bspt	 & 4	 & 262	 & 0	 & 264	 & 1	 & 0.7  & -22\% \\
 & \multicolumn{7}{c}{$\ldots{}$ plus 56 additional programs $\ldots{}$} \\
sphere &	4	 & 343	 & 0	 & 366	 & 2  & 0.7	 & -45\% \\
symalg & 	5	 & 402	 & 0	 & 453	 & 64  & 1.0	 & -32\% \\
x2n1	& 4 & 	345	 & 0	 & 385	 & 2  & 0.8	 & -57\% \\
\vspace{-1ex} \\
\multicolumn{8}{@@{}l}{Summary of all other programs:} \\
Minimum         & 2 & 60   & 0 & 46 &  0 & 0.1  & -78\% \\
Maximum         & 14 & 580 & 1 & 581 & 100 & 1.2  & 27\% \\
Average         & 5 & 260  & 0 & 232 & 5 & 0.5  & -30\% \\
\hline
\end{tabular*}
\end{table}

We have tested our method with programs drawn from the nofib benchmark suite \cite{nofib}, and the results are given in Table \ref{tab:results}. Looking at the input Core programs, we see many sources of functional values.

\begin{itemize}
\item Type classes create dictionaries which are implemented as tuples of functions.
\item The monadic bind operation is higher-order.
\item The IO data type is implemented as a function.
\item The Haskell |Show| type class uses continuation-passing style extensively.
\item List comprehensions in Yhc are desugared to continuation-passing style. There are other translations which require less functional value manipulations \cite{coutts:stream_fusion}.
\end{itemize}

We have tested all 14 programs from the imaginary section of the nofib suite, 35 of the 47 spectral programs, and 17 of the 30 real programs. The remaining 25 programs do not compile using the Yhc compiler, mainly due to missing or incomplete libraries. Upon applying our defunctionalisation method, 4 programs are curtailed by the termination bound, and 5 additional programs contain expressions which create functional values. Following the proof in \S\ref{sec:completeness}, the remaining 57 programs are guaranteed not to create any functional values at runtime. We first discuss the residual programs containing the creation of functional values, then make some observations about each of the columns in the table.

\subsection{Higher-Order Residues}

All four programs curtailed by the termination bound are listed in Table \ref{tab:results}. The lift program uses pretty-printing combinators, while the other three programs use parser combinators. In all programs, the combinators are used to build up a functional value representing the action to perform, storing an unbounded amount of information inside the functional value, which therefore cannot be removed.

The five programs that are not curtailed by the termination bound, but still contain residual higher-order expressions, are as follows:

\begin{example}
\textit{The integer and maillist programs} pass functional values to primitive functions. The maillist program calls the |catch| function (see \S\ref{sec:primitives}). The integer program passes functional values to the |seq| primitive, using the following function:

\begin{code}
seqlist []      = return ()
seqlist (x:xs)  = x `seq` seqlist xs
\end{code}

This function is invoked with the IO monad, so the |return ()| expression is a functional value. It is impossible to remove this functional value without having access to the implementation of the |seq| primitive.
\end{example}

\begin{example}
\textit{The pretty, constraints and mkhprog programs} pass functional values to expressions that evaluate to $\bot$. The case in pretty comes from the fragment:

\begin{comment}
% for paper haskell
\begin{code}
data PrettyRep = PrettyRep
ppBeside :: Pretty -> Pretty -> Pretty
\end{code}
\end{comment}
\begin{code}
type Pretty = Int -> Bool -> PrettyRep

ppBesides :: [Pretty] -> Pretty
ppBesides = foldr1 ppBeside
\end{code}

Here |ppBesides xs| evaluates to $\bot$ if |xs == []|. The $\bot$ value will be of type |Pretty|, and will be given further arguments, which can be functional arguments. In reality, the code ensures that the input list is never |[]|, so the program will never fail with this error.
\end{example}

\subsection{Termination Bound}

The termination bound used varies from 2 to 11 for the sample programs (see Bound in Table \ref{tab:results}). If we exclude the integer program, which is complicated by the primitive operations on functional values, the highest bound is 8. Most programs have a termination bound of 4. There is no apparent relation between the size of a program and the termination bound.

\subsection{Creating of Functional Values}

We use Yhc generated programs as input, which have been lambda lifted \cite{lambda_lift}, so contain no lambda expressions. The residual program has no partial application, only lambda expressions. Most programs in our test suite start with hundreds of partial applications, but only 5 residual programs contain lambda expressions (see HO Create in Table \ref{tab:results}).

For the purposes of testing defunctionalisation, we have worked on unmodified Yhc libraries, including all the low-level detail. For example, |readFile| in Yhc is implemented in terms of file handles and pointer operations. Most analysis operations work on an abstracted view of the program, which reduces the number and complexity of functional values.

\subsection{Uses of Functional Values}

While very few programs have residual functional values, a substantial number make use of general application, and use over-application of functions (see HO Use in Table \ref{tab:results}). In most cases these result from supplying |error| calls with additional arguments, typically related to the desugaring of |do| notation and pattern matching within Yhc.

\subsection{Execution Time}
\label{sec:time}

The timing results were all measured on a 1.2GHz laptop, running GHC 6.8.2 \cite{ghc}. The longest execution time was just over one second, with the average time being half a second (see Time in Table \ref{tab:results}). The programs requiring most time made use of floating point numbers, suggesting that library code requires most effort to defunctionalise. If abstractions were given for library methods, the execution time would drop substantially.

In order to gain acceptable speed, we perform a number of optimisations over the algorithm presented in \S\ref{sec:detailed}. (1) We transform functions in an order determined by a topological sort with respect to the call-graph. (2) We delay the transformation of dictionary components, as these will often be eliminated. (3) We track the arity and boxed lambda status of each function.

\subsection{Program Size}

We measure program size by counting the number of lines of Core code. On average the size of the resultant program is smaller by 30\% (see Size in Table \ref{tab:results}). The decrease in program size is mainly due to the elimination of dictionaries holding references to unnecessary code. An optimising compiler will perform dictionary specialisation, and therefore is likely to also reduce program size. We do not claim that defunctionalisation reduces code size, merely hope to alleviate concerns raised by previous papers \cite{chin:higher_order_removal} that doing so might cause an explosion in code size.


\section{Related Work}
\label{sec:related}

\subsection{Reynolds style defunctionalisation}

Reynolds style defunctionalisation \cite{reynolds:defunc} is the seminal method for generating a first-order equivalent of a higher-order program.

\begin{example}
\begin{code}
map f []      = []
map f (x:xs)  = f x : map f xs
\end{code}

\noindent Reynolds' method works by creating a data type to represent all values that |f| may take anywhere in the whole program. For instance, it might be:

\ignore\begin{code}
data Function = Head | Tail

apply Head  x = head  x
apply Tail  x = tail  x

map f []      = []
map f (x:xs)  = apply f x : map f xs
\end{code}

\noindent Now all calls to |map head| are replaced by \ignore|map Head|.
\end{example}

Reynolds' method works on all programs. Defunctionalised code is still type safe, but type checking would require a dependently typed language. Others have proposed variants of Reynolds' method that are type safe in the simply typed lambda calculus \cite{bell:type_driven_defunctionalization}, and within a polymorphic type system \cite{pottier:polymorhpic_typed_defunctionaization}.

The method is complete, removing all possible higher-order functions, and preserves space and time behaviour. The disadvantage is that the transformation essentially embeds a mini-interpreter for the original program into the new program. The control flow is complicated by the extra level of indirection and in practice the |apply| interpreter is a bottleneck for analysis. Various analysis methods have been proposed to reduce the size of the |apply| function, by statically determining a safe subset of the possible functional values at a call site \cite{cejtin:closure_conversion,grin}.

Reynolds' method has been used as a tool in program calculation \cite{danvy:defunctionalization_at_work,hutton:calculating_an_exceptional_machine}, often as a mechanism for removing introduced continuations. Another use of Reynolds' method is for optimisation \cite{jhc}, allowing flow control information to be recovered without the complexity of higher-order transformation.

\subsection{Removing Functional Values}

The closest work to ours is by \citet{chin:higher_order_removal}, which itself is similar to that of \citet{nelan:firstification}. They define a higher-order removal method, with similar goals of removing functional values from a program. Their work shares some of the simplification rules, the arity raising and function specialisation. Despite these commonalities, there are big differences between their method and ours.

\begin{itemize}
\item Their method makes use of the \textit{types} of expressions, information that must be maintained and extended to work with additional type systems.
\item Their method has \textit{no inlining} step, or any notion of boxed lambdas. Functional values within constructors are ignored. The authors suggest the use of deforestation \cite{wadler:deforestation} to help remove them, but deforestation transforms the program more than necessary, and still fails to eliminate many functional values.
\item Their specialisation step only applies to outermost lambda expressions, not lambdas within constructors.
\item To ensure termination of the specialisation step, they \textit{never specialise a recursive function} unless it has all functional arguments passed identically in all recursive calls. This restriction is satisfied by higher-order functions such as |map|, but fails in many other cases.
\end{itemize}

In addition, functional programs now use monads, IO continuations and type classes as a matter of course. Such features were still experimental when Chin and Darlington developed their method and it did not handle them. Our work can be seen as a successor to theirs, indeed we achieve most of the aims set out in their future work section. We have tried their examples, and can confirm that all of them are successfully handled by our system. Some of their observations and extensions apply equally to our work: for example, they suggest possible methods of removing accumulating functions such as in Example \ref{ex:functional_lists}.

\subsection{Partial Evaluation and Supercompilation}

The specialisation and inlining steps are taken from existing program optimisers, as is the termination strategy of homeomorphic embedding. A lot of program optimisers include some form of specialisation and so remove some higher-order functions, such as partial evaluation \cite{jones:partial_evaluation} and supercompilation \cite{supercompilation}. We have certainly benefited from ideas in both these areas in developing our algorithms. Our initial attempt at removing functional values involved modifying a supercompiler \cite{me:supero}. But the optimiser is not attempting to preserve correspondence to the original program, so will optimise all aspects of the program equally, instead of focusing on the higher-order elements. Overall, the results were poor.

\section{Conclusions and Future Work}
\label{sec:conclusion}

Higher-order functions are very useful, but may pose difficulties for certain types of analysis. Using the method we have described, it is possible to remove most functional values from most programs. A user can still write higher-order programs, but an analysis tool can work on equivalent first-order programs.

Our method has already found practical use within the Catch tool, and we hope it can be of benefit to others. We have released our tool, both as a command line program, and as a library that analysis programs can invoke. Currently the implementation is based around the core language from the Yhc compiler, which restricts our input programs to the Haskell 98 language. By making use of the GHC front end, we could deal with many language extensions.

Our method is whole program, requiring sources for all function definitions. This requirement both increases transformation time, and precludes the use of closed source libraries. We may be able to relax this requirement, precomputing first-order variants of libraries, or permitting some components of the program to be ignored.

We have developed our method for analysis, not performance. However, for many simple examples, the resultant program performs better than the original. By restricting rules that reduce sharing, our defunctionalisation method may be appropriate for integration into an optimising compiler.

The use of a numeric termination bound in the homeomorphic embedding is regrettable, but practically motivated. We need further research to determine if such a numeric bound is necessary, or if other measures could be used.

Many analysis methods, in fields such as strictness analysis and termination analysis, start out first-order and are gradually extended to work in a higher-order language. Defunctionalisation offers an alternative approach, instead of extending the analysis method, we transform the functional values away, enabling more analysis methods to work on a greater range of programs.


\bibliographystyle{plainnat}
\bibliography

\end{document}
